\chapter{结论与展望}
\section{本文工作总结}
随着计算机处理的数据量不断攀升,集中式存储系统逐渐被分布式存储系统所取代。然而在分布式存储系统中,由于采用大量低成本的节点共同进行数据的存储,因此节点发生损坏的风险大大增加,数据的冗余备份与恢复便成为了目前分布式存储系统研究的热点问题之一。本文通过对现有的分布式文件系统中的数据冗余和备份方案做出总结,明确了基于纠删码的数据冗余方案相比于多副本备份方案可以大大降低数据冗余度。紧接着通过对纠删码编码算法和纠删码在分布式存储系统中研究成果的分析,针对基于纠删码的分布式文件系统中编码策略方面和数据修改方面存在的问题,提出了一种基于纠删码和对象存储的分布式文件系统的实现方案,此方案主要分为以下两大部分:

针对不同数据具有不同可用性,且不同磁盘具有不同可靠性的特点,我们通过改进当前分布式文件系统中纠删码编码方案均为事先确定的方案,提出了 On-demand ARECS 算法,动态地选择编码策略和存储节点。On-demand ARECS 算法首先通过磁盘的 S.M.A.R.T. 信息得到磁盘的重分配扇区数,并根据磁盘重分配扇区数的统计值估计磁盘的故障率。通过对存储于分布式存储系统中对象的组合情况分析,我们进一步推导得到了对应的对象可用性。针对数据中心中损坏磁盘的替换与数据重建,我们通过在模型中增加数据重建时间进行反应。在满足基于对象重要性确定的可用性约束的磁盘组中,我们通过最小化数据传输和存储延迟的方法动态选取最优的存储磁盘组。针对大文件直接进行纠删码编码时编码矩阵过大导致的运算效率低下的问题,我们提出了数据切分为多块分别存储时的节点选择模型。我们在 Tahoe-LAFS 分布式文件系统上实验了 On-demand ARECS 算法,相对于目前的纠删码编码策略,On-demand ARECS 算法将存储空间冗余平均降低了 $18\%$,并将存储系统的文件传输时间降低了 $46\%$,大大提高了分布式存储系统的存储空间利用率,并改善了分布式存储系统的存储效率。

针对基于纠删码的分布式存储系统在修改数据时需要重新编解码,导致数据修改的代价较高,难以用于可读写的数据存储的特点,本文提出了使用修改事件代替直接修改对象的方法。我们介绍了基于纠删码的分布式对象存储系统的存储结构,并设计了一个基于分布式对象存储的文件系统中间件,通过将修改事件内容单独存储为修改数据对象,并在分布式文件系统的元数据中对其进行索引的方法减少数据的解码和重新编码运算。针对元数据可能存在频繁读写的情况,我们设计了基于数组的无锁队列结构对元数据进行更新,大大提高了系统的并发性能。针对此方法在数据存在大量修改记录的极端情况中,读取时最坏情况遍历比较次数可能过多的特点,设计了在系统空闲时进行对象数据合并的算法。我们在基于 On-demand ARECS 算法的 Tahoe-LAFS 分布式文件系统上实验了此文件修改算法,结果表明,基于修改事件的分布式文件存储算法大大降低了基于纠删码引擎的分布式文件系统中修改或附加数据所用的时间,数据的修改和附加操作平均耗时仅为原来的 $30\%\sim40\%$,提升了基于纠删码的分布式文件系统在通用存储领域的可用性。
\section{未来研究展望}
尽管在分布式文件系统中测试的实验数据表明,上述方法在基于纠删码对象存储的分布式文件系统中的文件存储速度方面,相比于传统的方法有了很大改善,但距离基于纠删码的分布式存储系统完全替代基于多副本备份的分布式存储系统的目标,还有许多问题需要解决。我们通过总结,认为在当前基于纠删码的分布式文件系统中,可以通过以下几个方面加以改进:

一是在数据损坏时的数据重建问题。我们目前的研究主要基于数据存储时的性能开销,在面对存储设备损坏时,基于纠删码的存储方案需要通过矩阵计算恢复原始数据,相比于多副本备份方案,需要对原始数据进行解码和对冗余数据进行编码,进而消耗更多的网络带宽资源和节点计算资源。由于在分布式存储系统中节点损坏事件时有发生,数据重建所占用的计算和网络资源在数据中心中不能被忽视。未来的研究如果可以通过减小纠删码编码和解码的开销或优化重建传输策略等方式,更加高效地进行基于纠删码存储的分布式文件系统的数据重建,从而更大限度地减少数据中心内部的数据传输和节点的计算,将可以使极大地降低基于纠删码的分布式存储系统的额外性能开销。

二是纠删码编码和解码的开销问题。虽然目前的计算机系统已经能有效地计算纠删码的编码和解码,但相比于多副本备份的直接读取,对于热点数据的多次编码和解码还是有着较大的性能开销。此额外开销主要表现在数据写入时进行冗余数据编码的计算开销,在数据读取时通过冗余数据进行原始数据恢复的开销,以及在存储设备损坏时数据重建时原始数据的解码以及可能的冗余数据编码的开销等等。因此,研究如何通过改进纠删码编码方案,设计新的纠删码编码矩阵减少计算量,增加算法并行度并通过 GPU 或 ASIC 专用硬件等方案进行大规模的并行计算,实现更低代价的纠删码编码和解码,从而大大提升基于纠删码编码的分布式文件系统的数据写入、读取和重建性能,仍是分布式存储系统需要研究的重要问题。

三是磁盘故障率的有效估计问题。本文基于估计的磁盘故障率进行后续的编码方案选择,但通过引入硬盘 S.M.A.R.T. 信息,读取磁盘的重分配扇区数,并通过数据中心中大量磁盘的重分配扇区数统计量来估算磁盘的故障率的方法可能并不够准确。由于磁盘的重分配扇区数仅仅是反映硬盘健康程度的指标之一,这种估计的置信度可能并不够高,若能综合通过磁盘多种维度的信息进行磁盘故障率的估计,则磁盘故障率的估计值能更加准确。此外,部分存储介质可能不存在重分配扇区的操作,如何进行这部分介质的故障率估计也是需要研究的问题之一。若能更加准确的预计磁盘发生损坏的概率,则可以使本文估算的数据可用性置信度更高,进而提升存储空间利用率。
